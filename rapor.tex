%
% Tasarım projesi raporu için hazırlanmış latex şablonu. Bu şablon
% 2022-2023 eğitim öğretim yılları arasında son sınıf
% öğrencisi Berkay Çubuk <berkay@berkaycubuk.com>
% tarafından Bursa Uludağ Üniversitesi proje raporlarında
% kullanılması için hazırlanmıştır.
%
% Bu şablonunun kullanımı kendi sorumluluğunuzdadır.
% Şablon tasarımcısı herhangi bir sorumluluk kabul etmemektedir.
%
\documentclass[hidelinks,12pt]{article}
% türkçe desteği
\usepackage[turkish,shorthands=off]{babel}
\usepackage[utf8]{inputenc}
\usepackage[T1]{fontenc}

% linkler
\usepackage{hyperref}

% kod görüntüleme
\usepackage{minted}
\usepackage{xcolor}
\definecolor{LightGray}{gray}{0.9}

% başlık fontları
\usepackage{sectsty}
\sectionfont{\fontsize{12}{15}\selectfont}
\subsectionfont{\fontsize{12}{15}\selectfont}
\subsubsectionfont{\fontsize{12}{15}\selectfont}

% diagramlar
\usepackage{tikz}
\usepackage{smartdiagram}

% atıf yapma
\usepackage[
        backend=biber,
        style=authoryear,
        sorting=nyt,
    ]{biblatex}
 \addbibresource{./bib.bib}

\usepackage{mathptmx}
\usepackage{indentfirst}
\usepackage{float}
\usepackage[T1]{fontenc}
\usepackage{graphicx}
\usepackage{geometry}
\usepackage{multicol}
\geometry{
    a4paper,
    left=4cm,
    top=3cm,
    bottom=3cm,
    right=2cm,
}
\usepackage{setspace}
\onehalfspacing

\usepackage{fancyhdr}
\fancyhf{}
\fancyhead[C]{\thepage}
\renewcommand{\headrulewidth}{0pt}
\pagestyle{fancy}
\counterwithin{figure}{section}

\renewcommand*{\nameyeardelim}{\addcomma\space}

% \renewcommand{\figurename}{Şekil}
% \renewcommand{\tablename}{Tablo}

\usepackage{titlesec}

\titlelabel{\thetitle.\quad}

\titlespacing*
    {\section}
    {20pt}
    {3.5ex plus 1ex minus .2ex}
    {2.3ex plus .2ex}

\titlespacing*
    {\subsection}
    {20pt}
    {3.5ex plus 1ex minus .2ex}
    {2.3ex plus .2ex}

\titlespacing*
    {\subsubsection}
    {20pt}
    {3.5ex plus 1ex minus .2ex}
    {2.3ex plus .2ex}

% doküman bilgileri
\author{Ad Soyad\\Öğrenci Numarası}
\title{Rapor Başlığı}
\date{}

% içindekiler bölümünün başlığını yok etmek için
\makeatletter
\renewcommand\tableofcontents{%
    \@starttoc{toc}%
}
\makeatother
\makeatletter
\renewcommand*\listfigurename{}

\begin{document}

% kapak sayfası
\begin{titlepage}
    \begin{tabular}{c c c}
        \includegraphics[width=3cm]{./bursa_uludag_logo.png} &
        T.C. &
        \includegraphics[width=3cm]{./bursa_uludag_muh_logo.png} \\
        & BURSA ULUDAĞ ÜNİVERSİTESİ & \\
        & MÜHENDİSLİK FAKÜLTESİ & \\
        & BİLGİSAYAR MÜHENDİSLİĞİ BÖLÜMÜ & \\
    \end{tabular}
    \newline
    \newline
    \newline
    \newline
    \newline
    \newline
    \begin{center}
        \large{Proje Başlığı}
    \end{center}
    \hfill \break
    \newline
    \begin{center}
        Ad Soyad\\Öğrenci Numarası\\
    \end{center}
    \hfill \break
    \newline
    \begin{center}
        TASARIM PROJESİ FİNAL RAPORU\\
    \end{center}
    \hfill \break
    \newline
    \newline
    \newline
    \newline
    \newline
    \newline
    \hfill \break
    \begin{center}
        Bursa 2023
    \end{center}
\end{titlepage}

\newpage
\thispagestyle{empty}
    \begin{center}
    T.C. \\
    \end{center}
    \begin{center}
    BURSA ULUDAĞ ÜNİVERSİTESİ \\
    \end{center}
    \begin{center}
    MÜHENDİSLİK FAKÜLTESİ \\
    \end{center}
    \begin{center}
    BİLGİSAYAR MÜHENDİSLİĞİ BÖLÜMÜ \\
    \end{center}
    \hfill \break
    \newline
    \newline
    \newline
    \begin{center}
        \large{Proje Başlığı}
    \end{center}
    \hfill \break
    \newline
    \begin{center}
        Ad Soyad\\Öğrenci Numarası\\
    \end{center}
    \hfill \break
    \newline
    \newline
    \newline
    \newline
    \begin{center}
        \textbf{Proje Danışmanı} \\ Proje Danışmanı Ad Soyad
    \end{center}
    \hfill \break
    \newline
    \newline
    \newline
    \newline
    \newline
    \newline
    \newline
    \hfill \break
    \begin{center}
        Bursa 2023
    \end{center}
\newpage

% özet sayfası
\pagenumbering{roman}
\setcounter{page}{2}
\begin{center}
\section*{\centering ÖZET}
\end{center}
\addcontentsline{toc}{section}{\protect\numberline{}ÖZET}
\quad Özet içeriği

\newpage

% abstract sayfası
\begin{center}
\section*{\centering ABSTRACT}
\end{center}
\addcontentsline{toc}{section}{\protect\numberline{}ABSTRACT}
\quad Abstract içeriği

\newpage

% içindekiler sayfası
\newpage
\thispagestyle{empty}
    \section*{\centering İÇİNDEKİLER}
    \tableofcontents
\newpage

% şekiller dizini sayfası
\section*{\centering ŞEKİLLER DİZİNİ}
\addcontentsline{toc}{section}{\protect\numberline{}ŞEKİLLER DİZİNİ}
\listoffigures
\newpage

% giriş sayfası
\section{GİRİŞ}
\pagenumbering{arabic}

Giriş bölümünde konunun önemi, güncelliği, yapılacak çalışmanın amaçları ve
çalışmaya ilişkin genel bilgiler yer almalıdır.
\newpage

% kaynak araştırması sayfası
\section{KAYNAK ARAŞTIRMASI}

Kaynak araştırması bölümünde, konu ile
ilgili önceki çalışmalar, ulaşılan sonuçları vurgulamak üzere özetlenmelidir.
\newpage

% materyal ve yöntem sayfası
\section{MATERYAL ve YÖNTEM}

Materyal
ve yöntem bölümünde, proje çalışmasında kullanılan veri ve bilgiler, bunların elde
edilme yöntemleri, araştırma ve incelemede izlenen yöntem veya yöntemler hakkında
açıklamalar yapılmalıdır.
\newpage

% araştırma sonuçları sayfası
\section{ARAŞTIRMA SONUÇLARI}

Araştırma sonuçları bölümünde, araştırma kapsamına uygun
olarak elde edilen bilgi ve bulgular sunulmalıdır.
\newpage

% tartışmma sayfası
\section{TARTIŞMA}

Tartışma bölümünde ise, proje
çalışmasında elde edilen sonuçlar, önceki çalışma sonuçları ile karşılaştırılarak
çalışmaya ilişkin sonuçlar çıkartılmalı ve genel değerlendirme yapılmalıdır.
\newpage

% ekler sayfası
\section{EKLER}
Bitirme
projesi içinde yer almaları halinde konuyu dağıtıcı ve okumada sürekliliği engelleyici
nitelikteki uzun açıklamalar, bilgisayar programları, örnek hesaplamalar gibi bilgiler
veya paftalar vb. bilgiler EKLER bölümünde sunulmalıdır. Bunlar; geniş ve ayrıntılı
tablolar, anket formları, belgeler, geniş haritalar ve benzerleridir. Bu bölümde yer
alacak her bir belge ya da açıklama için bir başlık seçilmeli ve bunlar sunuş sırasına
göre Ek 1., Ek 2. gibi her biri ayrı bir sayfadan başlayacak şekilde numaralandırılarak
sunulmalıdır.
\newpage

% kaynaklar sayfası
\newpage
\section{KAYNAKLAR}
\printbibliography[heading=none]

\newpage

% teşekkür sayfası
\section{TEŞEKKÜR}

\begin{flushright}
    Ad Soyad\\
    Bursa, 2023
\end{flushright}

Projeyi hazırlayanın belirtmek istediği özel mesaj durumunda olup konu
hakkındaki kişisel görüş, amaç ve dileklerini kapsamalıdır. Projeyi destekleyen
kuruluşlar varsa, bunlardan söz edilebilir ve istenirse, ilgililere teşekkür edinilebilir.
Sayfanın üst kısmına, paragraftan başlayacak şekilde büyük harflerle (kalın ve koyu
karakter) TEŞEKKÜR yazılmalıdır. Teşekkür başlığının sağ alt kısmında yazarın ismi
ve soyadı yazılmalıdır. Yazar adının hemen altına projenin yapıldığı il ve yıl
yazılmalıdır.
\newpage

% özgeçmiş sayfası
\section{ÖZGEÇMİŞ}

\subsection{Ad Soyad}
Bitirme Projesini hazırlayan öğrencinin kısa özgeçmişi bu bölümde yer almalıdır.

\end{document}
